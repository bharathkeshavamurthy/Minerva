%
%  revised  front.tex  2017-01-08  Mark Senn  http://engineering.purdue.edu/~mark
%  created  front.tex  2003-06-02  Mark Senn  http://engineering.purdue.edu/~mark
%
%  This is ``front matter'' for the thesis.
%
%  Regarding ``References'' below:
%      KEY    MEANING
%      PU     ``A Manual for the Preparation of Graduate Theses'',
%             The Graduate School, Purdue University, 1996.
%      PU8    ``A Manual for the Preparation of Graduate Theses'',
%             Eighth Revise Edition, Purdue University.
%      TCMOS  The Chicago Manual of Style, Edition 14.
%      WNNCD  Webster's Ninth New Collegiate Dictionary.
%
%  Lines marked with "%%" may need to be changed.
%

  % Statement of Thesis/Dissertation Approval Page
  % This page is REQUIRED.  The page should be numbered page ``ii''
  % and should NOT be listed in your TABLE OF CONTENTS.
  % References: PU8 ordinal pages 5 and 29.
  % The web page https://engineering.purdue.edu/AAE retrieved on
  % January 8, 2017 had "School of Aeronautics and Astronautics"---that
  % is used instead of "Department of Aeronautics and Astronautics"
  % below.
\begin{statement}
  \entry{Dr.~Nicol\`{o} Michelusi, Chair}{School of Electrical and Computer Engineering}
  \entry{Dr.~Xiaojun Lin}{School of Electrical and Computer Engineering}
  \entry{Dr.~Shreyas Sundaram}{School of Electrical and Computer Engineering}
  \approvedby{Dr.~Dimitrios Peroulis}{Head of the School of Electrical and Computer Engineering}
\end{statement}

 % Acknowledgements page is optional but most theses include
 % a brief statement of appreciation or recognition of special
 % assistance.
 % Reference: PU 16.
\begin{acknowledgments}
 This research has been funded in part by NSF, under grant CNS-1642982.
\end{acknowledgments}

  % The Table of Contents is required.
  % The Table of Contents will be automatically created for you
  % using information you supply in
  %     \chapter
  %     \section
  %     \subsection
  %     \subsubsection
  % commands.
  % Reference: PU 16.
\tableofcontents

  % If your thesis has figures, a list of figures is required.
  % The List of Figures will be automatically created for you using
  % information you supply in
  %     \begin{figure} ... \end{figure}
  % environments.
  % Reference: PU 16.
\listoffigures

  % Abstract is required.
  % Note that the information for the first paragraph of the output
  % doesn't need to be input here...it is put in automatically from
  % information you supplied earlier using \title, \author, \degree,
  % and \majorprof.
  % Reference: PU 17.
\begin{abstract}
Cognitive Radio technologies have been touted to be instrumental in solving resource-allocation problems in resource-constrained radio environments. The adaptive computational intelligence of these radios facilitates the dynamic allocation of network resources\texttt{-{}-}particularly, the spectrum, a scarce physical asset. In addition to consumer-driven innovation that is governing the wireless communication ecosystem, its associated infrastructure is being increasingly viewed by governments around the world as critical national security interests\texttt{-{}-}the US Military instituted the DARPA Spectrum Collaboration Challenge which requires competitors to design intelligent radios that leverage optimization, A.I., and game-theoretic strategies in order to efficiently access the RF spectrum in an environment wherein every other competitor is vying for the same limited resources. In this work, we detail the design of our radio, i.e., the design choices made in each layer of the network protocol stack, strategies rigorously derived from convex optimization, the collaboration API, and heuristics tailor-made to tackle the unique scenarios emulated in this DARPA Grand Challenge. We present performance evaluations of key components of our radio in a variety of military and disaster-relief deployment scenarios that mimic similar real-world situations. Furthermore, specifically focusing on channel access in the MAC, we formulate the spectrum sensing and access problem as a POMDP; derive an optimal policy using approximate value iteration methods; prove that our strategy outperforms the state-of-the-art, and facilitates means to control the trade-off between secondary network throughput and incumbent interference; and evaluate this policy on an ad-hoc distributed wireless platform constituting ESP32 radios, in order to study its implementation feasibility.
\end{abstract}