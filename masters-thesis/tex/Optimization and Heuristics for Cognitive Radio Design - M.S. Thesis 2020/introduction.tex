%
%  revised  introduction.tex  2011-09-02  Mark Senn  http://engineering.purdue.edu/~mark
%  created  introduction.tex  2002-06-03  Mark Senn  http://engineering.purdue.edu/~mark
%
%  This is the introduction chapter for a simple, example thesis.
%


\chapter{INTRODUCTION}\label{A}
"Wireless Companies share the airwaves"\texttt{-{}-}an article \cite{WSJ:CBRS} published in the Wall Street Journal in December 2019 emphasises a few key points about Cognitive Radio (CR) technologies, also known as Dynamic Spectrum Access (DSA) or neXt-Generation (xG) technologies in the wireless communication landscape: firstly, it reiterates a critical fact that has long been known in the industry that the economics of spectrum licensing plays a vital role in driving innovation in Radio Access Technologies (RATs); secondly, the article reports that in September 2019, the Federal Communications Commission (FCC) allowed private players in the telephone, cable, and internet space to provide their services over Citizens Broadband Radio Service (CBRS) without having to buy a license, provided their transmissions do not interfere with the U.S. Navy and entities that pay for a license; thirdly, it details the three tiers under which the CBRS spectrum ($3550$MHz-$3700$MHz) has been categorized by the FCC\texttt{-}the U.S. military (particularly, naval radar operators and aircraft communications), "Priority-Access" licensees constituting service providers that pay for access to this spectrum, and "General Authorized Access" users comprising unlicensed users; and finally, the article reports on the administrative and bureaucratic red-tape that prevented this policy that was first brought-up in 2012 to be realized almost 8 years later, quoting Craig Moffett, an analyst at MoffettNathanson LLC\texttt{-}``The concept has been in place for a really long time, waiting for the pieces to fall in place".

With fifth-generation (5G) mobile communication technologies in full-deployment mode around the world today, several countries\texttt{-{}-}especially, the U.S. and China, are vying to dominate the space\texttt{-}with the U.S. being extra cautious, citing national security concerns. Many analysts have expressed the need for better spectrum auctioning and availability provisions at the federal level in order to facilitate efficient deployment of 5G services across the country \cite{WSJ:5Gdominance}. The 5G ecosystem brings in an extraordinary demand for these limited spectrum resources due to the promises of extremely high data rates; extremely low latencies; high reliability for critical infrastructure; massive Machine-Type Communication (MTC) enabling the embedded-IoT boom involving Wireless Sensor Networks (WSNs)\texttt{-}both, industrial and personal, Human-Computer interfaces, autonomous vehicles, and Vehicular Ad-Hoc Networks (VANs); and improved mobility \cite{WCL:1,Ericsson:5Gusecases}. Although a significant number of research works exist in the state-of-the-art professing widespread adoption of cognitive radio technologies for the 5G landscape and trying to solve problems associated with shared access of spectrum resources \cite{WCL:7,WCL:6,WCL:4,WCL:5,WCL:9,WCL:10,WCL:11}, little progress has been made with respect to the real-world deployment of these technologies. Spectrum-sharing technologies have never been in the limelight more than they are today, especially among economists, academics, and national security analysts, with Holman Jenkins Jr. writing in the Journal, "...new spectrum-sharing technologies increasingly make the airwaves seem less scarce than once thought". He further goes on to state that efficient re-allocations of existing spectrum coupled with the widespread deployment spectrum-sharing technologies will have huge public benefits \cite{WSJ:HolmanJenkinsJr.}.

Exhibiting much-needed prescience, in 2016, a Grand Challenge was instituted by the U.S. Defense Advanced Research Projects Agency (DARPA), known as the Spectrum Collaboration Challenge (SC2) to understand the implications of spectrum sharing and to drive innovation in the space using Artificial Intelligence (A.I.). DARPA understood that the division of the spectrum into rigid, licensed bands is infeasible in the current wireless environment due to the massive demand \cite{DARPA:SC2}. The DARPA SC2 envisioned a fully collaborative radio environment in which competing radios exhibited collaborative spectrum access strategies to not only satisfy their individual Quality of Service (QoS) requirements, but to also view the problem altruistically, i.e., to allow for the entire ensemble to satisfy their QoS requirements. The DARPA SC2 involved several simulated scenarios that mimic similar situations these radios have to operate in, in the real-world, for example, troop-deployment scenarios in urban areas, high-priority audio and video communication among first responders fighting a wildfire, everyday user communication in consumer centers such as shopping malls, and scenarios in which the radios have to work around jammers and licensed users (incumbents) \cite{DARPA:SC2scenarios}. Cognitive Radio technologies typically involve solving spectrum sensing and access problems in an independent setting wherein the cognitive radio node uses its passive sensing capabilities to deliver its traffic over licensed bands, subject to constraints on the amount of interference caused to military and licensed users. While we do discuss our solution to the independent spectrum sensing and access problem in this work using tools from Dynamic Programming (DP) and estimation theory, the DARPA SC2 featured a more collaborative radio environment in which the radios deployed in certain scenarios communicated with each other using a shared protocol (i.e., language) over a common communication link (air link/public internet/satellite) in order to exchange relevant performance metrics and their respective QoS requirements, which would then be used in solving a joint resource-allocation problem for mutual benefits \cite{DARPA:SC2collaboration}. Leveraging the capabilities of Software-Defined Radios (SDRs) and A.I., competitors from both the industry and academia competed in the challenge that lasted for 3 years (Dec 2016-Oct 2019). The Purdue BAM! Wireless team from the Department of Electrical and Computer Engineering (ECE) designed radios from the ground-up for operations in Collaborative Intelligent Radio Network (CIRN) environments emulated in the DARPA Massive CHannel EMulator (MCHEM) known as the Colosseum. In this work, we detail the design principles underlying the development of our radios for the SC2, while also discussing their crucial performance metrics and behavioral characteristics. Various advancements are now being built-upon the standards and strategies established as a result of this Grand Challenge, including the CIRN Interaction Language (CIL), the Colosseum test-bed for public use, adaptive spectrum use visualization technologies, and A.I. techniques in various layers of the radio network protocol stack \cite{DARPASC2:end1,DARPASC2:end2,DARPASC2:end3,DARPASC2:end4}. Simplifying the problem by narrowing our focus on the design of optimal channel access strategies in a single cognitive radio node operating in a radio environment with multiple priority or licensed users, we introduce our POMDP formulation next.

From an independent cognitive radio perspective, our solution to the spectrum sensing and access problem in the Medium Access Control (MAC) layer of a cognitive radio node, referred to as a Secondary User (SU) from this perspective, sharing a discretized multi-channel AWGN radio environment with several licensed users, involves a Partially Observable Markov Decision Processes (POMDP) formulation \cite{WCL:paper}. As alluded to earlier, a cognitive radio facilitates efficient spectrum utilization by intelligently accessing unused spectrum holes across both time and frequency known as "spectrum white spaces", in order to deliver its network flows while limiting interference to the priority or licensed users (incumbents), referred to as Primary Users (PUs) from this perspective \cite{WCL:2}. In order to intelligently access these white spaces, the SU needs to solve for a channel sensing and subsequent access policy based on the noisy observations of the occupancy behavior of the PUs in the network. However, critical design limitations prevent the SU from sensing all the channels in the discretized spectrum of interest. These sensing limitations are primarily driven by energy efficiency requirements, with some additional restrictions imposed by the need for minimal sensing times \cite{WCL:3}. So, in view of these sensing limitations, the logical next step would be to develop algorithms that try to maximize the accuracy of incumbent occupancy behavior estimation, subject to upper bounds on the number of channels that can be sensed by the SU at the beginning of each time-slot: several works in the state-of-the-art \cite{WCL:4,WCL:5,WCL:6,WCL:7} propose algorithms to solve this limited information spectrum sensing and access problem. However, almost all of these works \cite{WCL:4,WCL:5,WCL:8,WCL:9,WCL:10,WCL:11} fail to leverage the correlations exhibited in the occupancy behavior of the incumbents across both frequency and time \cite{WCL:12}, which as we will illustrate later in this work, lead to significant improvements in the estimation accuracy, which in turn leads to a greater number of white spaces accessed by the SU for delivering its network flows, thereby resulting in a higher SU network throughput with lower levels of interference caused to the PUs in the network. In the sections that follow, we detail solutions to learn these frequency-time correlations in PU occupancy behavior, and to concurrently utilize these learned statistics to solve for an optimal sensing and access policy using approximate POMDP value iteration methods.

As alluded to earlier, the existing state-of-the-art does tackle the spectrum sensing and access problem, albeit with some underlying assumptions\texttt{-{-}-}many of these assumptions when broken down or generalized will lead to a better solution, as discussed in this work. Firstly, \cite{WCL:5} details a solution for distributed spectrum sensing employing SARSA with linear value function approximation. However, this work fails to capitalize on the correlated occupancy behavior of the PUs across channels. Additionally, the authors fail to provide a mechanism to manage the trade-off between secondary network throughput and incumbent interference, which we do. Unlike \cite{WCL:5}, although \cite{WCL:7} considers frequency correlation, the assumed observation model is noise-free, which is not realistic. On the other hand, we in this work, present a Hidden Markov Model (HMM) system-level framework in which the true occupancy states of the incumbents in the channels are hidden behind noisy observations at the SU's spectrum sensor\texttt{-{}-}HMMs call for the Viterbi algorithm (for state estimation), Baum-Welch algorithm (for model estimation), and POMDP formulations. In addition to the noise-free observation model assumptions in \cite{WCL:7}, our solution outperforms both the Minimum Entropy Merging (MEM) algorithms detailed in it, i.e., Markov Process Estimation coupled with Greedy Clustering (MPE-GC) and Markov Process Estimation (MPE) coupled with Minimum Entropy Increment Clustering (MPE-MEI). Additionally, among works that tackle this problem as an HMM framework \cite{WCL:6} like we do, the Viterbi algorithm is featured as a potential solution for occupancy behavior estimation. As illustrated in the subsequent sections of this work, not only does our solution outperform the Viterbi algorithm (with the same channel sensing limitations), our solution also provides for an online transition model estimation algorithm that operates concurrently with the approximate POMDP value iteration algorithm, i.e., PERSEUS. In contrast, the proposals outlined in \cite{WCL:6} and \cite{WCL:7} determine the time-frequency incumbent occupancy correlation structure offline using pre-loaded databases, which is inefficient in non-stationary settings.

In this section, we have detailed the need for spectrum sharing from an administrative, economic, and scientific perspective, which serves as the motivation for our work. In view of this need, we have hinted at the prescience of DARPA in establishing the SC2, the design of our radios specifically for this competition, and the technologies/standards to be born out of this Grand Challenge. Furthermore, narrowing our focus down to a sub-problem in cognitive radio design, i.e., spectrum sensing and access in the MAC layer of the radio, we have introduced the design of our solution along with a comparison, both in terms of the capabilities and the system performance, with other similar works in the state-of-the-art. Additionally, the subsequent sections of this work will present illustrations and metrics regarding the implementation of the optimal POMDP channel sensing and access policy on an ad-hoc distributed test-bed consisting of ESP32 radios, which will establish the simplicity in the policy's implementation.
