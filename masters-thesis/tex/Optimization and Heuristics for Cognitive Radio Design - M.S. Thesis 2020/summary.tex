%
%  summary.tex  2007-02-06  Mark Senn  http://www.ecn.purdue.edu/~mark
%

\chapter{SUMMARY}
Comprehending the need for spectrum sharing from an economic and national security point of view, in this work, we describe the various design principles involved in the development of cognitive radios, while specifically focusing on certain aspects of the design, proving their superiority to the state-of-the-art, and finally, their implementation feasibility. Detailing further, in this work, we describe the design principles underlying the operation of our cognitive radio in the DARPA SC2 Colosseum\texttt{-{}-}importantly, the CIRN architecture under which our networks were deployed in any given scenario, the OFDM PHY, the MCS adaptation algorithm driven by packet error rates, prioritized flow-scheduling involving a recursive-revisitation value-per-resource heuristic, the channel access algorithm driven by PSD measurements and collaboration messages, the CIL protocol, and multi-hop routing. Moreover, we present actual performance plots of these algorithms in DARPA SC2-emulated scenarios, in which the performance of our network is compared in real-time with that of other competitor networks deployed in the same scenario, in addition to an ensemble-view of the performance of all the networks in the scenario, to prove the need for collaboration among networks.

While acknowledging the possibility and potential of improvements in the other aspects of our radio's design, we focused our subsequent research particularly on the spectrum sensing and access algorithms in the MAC layer of the radio's network protocol stack and sought to improve it by adopting a rigorous mathematical approach as opposed to a more heuristic one incorporated in our DARPA SC2 radio. In this regard, in an AWGN observation model and a Raleigh fading channel model, we developed a POMDP formulation for channel sensing and access in a radio environment involving multiple incumbents exhibiting a time-frequency correlation structure in their occupancy behavior. In this POMDP formulation, in order to solve for the optimal channel sensing and access policy to be employed in the MAC layer of our radio, we designed an online parameter estimation algorithm leveraging tools from HMMs and MLE, and embedded it directly into a randomized point-based approximate value iteration method known as the PERSEUS algorithm with additional customizations such as fragmentation and belief update simplification. To prove the superiority of our POMDP framework for spectrum sensing and access, we have presented numerical evaluations of our algorithm against the state-of-the-art\texttt{-{}-}in doing so, we have proved that, for the same amount of deterioration in the throughput of the PUs in the network, our solution obtains a 37.5\% improvement in SU network throughput, compared to the MEM with MPE-MEI from \cite{WCL:7}; a 25\% improvement over a Neyman-Pearson Detector with no sensing restrictions from \cite{WCL:11}; and 6\% improvement over the Viterbi algorithm from \cite{WCL:6}. Critically, accounting for the need to handle deployment scenarios in which different interference constraints may be imposed at different times and in different geographical regions, our framework facilitates the trade-off between the SU network throughput and PU interference, by tuning the penalty parameter $\lambda$. Additionally, with this formulation, we drive home three crucial results: leveraging the time-frequency correlation structure exhibited in the occupancy behavior of the incumbents in the network leads to significant improvements in estimation accuracy, while allowing the radio to make wise decisions with limited information (due to channel sensing restrictions); adapting the spectrum sensing decisions according to past actions and their corresponding rewards leads to more white spaces being found for use by the cognitive radio; and an online EM algorithm for HMMs (known as the Baum-Welch algorithm) can be used to estimate the time-frequency occupancy correlation structure in tandem with a POMDP policy optimization algorithm, in non-stationary settings.

Finally, the performance metrics and illustrations presented from the POMDP optimal policy implementation experiment on an ad-hoc distributed network of ESP32 radios embedded on off-the-shelf e-puck2 robots, prove the simplicity in adapting the algorithms to different network setups and to different radio terminals having varying degrees of computational capabilities.