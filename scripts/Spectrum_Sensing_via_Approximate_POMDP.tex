\documentclass[10pt,twocolumn]{IEEEtran}
% \documentclass[12pt, draftcls, onecolumn]{IEEEtran}
\makeatletter
\def\subsubsection{\@startsection{subsubsection}
                                 {3}
                                 {\z@}
                                 {0ex plus 0.1ex minus 0.1ex}
                                 {0ex}
                             {\normalfont\normalsize\bfseries}}
\makeatother
\usepackage[T1]{fontenc}
\usepackage{subfigure}
\usepackage{ulem}
\usepackage{amsmath}
\allowdisplaybreaks
\usepackage{hhline}
\usepackage{graphicx}
\usepackage{yfonts,color}
\usepackage{soul,xcolor}
\usepackage{verbatim}
\usepackage{amsmath}
\allowdisplaybreaks
\usepackage{amssymb}
\usepackage{amsthm}
\usepackage{float}
\usepackage{bm}
\usepackage{url}
\usepackage{array}
\usepackage{cite}
\usepackage{tikz}
\usepackage{framed}
\usepackage{balance}
\usepackage{epsfig,epstopdf}
\usepackage{booktabs}
\usepackage{courier}
\usepackage{subfigure}
\usepackage{pseudocode}
\usepackage{enumerate}
\usepackage{algorithm}
\usepackage{algpseudocode}
\newtheorem{definition}{Definition}
\newtheorem{theorem}{Theorem}
\newtheorem{lemma}[theorem]{Lemma}
\newtheorem{proposition}[theorem]{Proposition}
\newtheorem{corollary}[theorem]{Corollary}
\newtheorem{assumption}{Assumption}
\newtheorem{remark}{Remark}
\renewcommand{\algorithmicrequire}{\textbf{Initialization:}}  
\renewcommand{\algorithmicensure}{\textbf{Output:}}  
\newcommand{\rom}[1]{\uppercase\expandafter{\romannumeral #1\relax}}
\usepackage{color}
\usepackage{soul,xcolor}
\newcommand{\sst}[1]{\st{#1}}
%\newcommand{\sst}[1]{}
\newcommand{\nm}[1]{{\color{blue}\bf{[NM: #1]}}}
%\newcommand{\nm}[1]{}
\newcommand{\bk}[1]{{\color{magenta}{[BK: #1]}}}
\newcommand{\nmmath}[1]{{\color{blue}\text{\bf{[NM: #1]}}}}
\newcommand{\gs}[1]{{\color{orange}\bf{[GS: #1]}}}
\newcommand{\remove}[1]{{\color{magenta}{\bf REMOVE: [#1]}}}
\DeclareMathOperator*{\argmax}{arg\,max}
\DeclareMathOperator*{\argmin}{arg\,min}
\usepackage{cancel}
\newcommand\mst[2][red]{\setbox0=\hbox{$#2$}\rlap{\raisebox{.45\ht0}{\textcolor{#1}{\rule{\wd0}{2pt}}}}#2} 
%\newcommand\mst[2][red]{} 
\newcommand{\add}[1]{{\color{red}{#1}}}
\newcommand{\ull}[1]{\textbf{\color{red}\ul{#1}}}
\renewcommand{\baselinestretch}{0.96}
\normalem
\title{Spectrum Sensing in Cognitive Radio Networks
\\
via Approximate POMDP}
\author{Bharath Keshavamurthy, Nicol\`{o} Michelusi
\thanks{This research has been funded in part by NSF under grant CNS-1642982.}
\thanks{The authors are with the School of Electrical and Computer Engineering, Purdue University. email: \{bkeshava,michelus\}@purdue.edu.}
\vspace{-12mm}}
\begin{document}
\maketitle
\thispagestyle{empty}
\pagestyle{empty} 
\setulcolor{red}
\setul{red}{2pt}
\setstcolor{red}
\begin{abstract}
In this paper, a novel spectrum sensing and access strategy based on POMDPs is proposed. A cognitive radio learns the correlation models defining the occupancy behavior of incumbents, based on which it devises an optimal spectrum sensing and access policy. The optimization complexity is ameliorated via point-based value iteration methods. Numerical evaluations demonstrate that our framework achieves higher SU throughput with lower PU interference, compared to clustering algorithms from the state-of-the-art and a Neyman-Pearson detector that assumes independence among channels. Furthermore, our scheme achieves the throughput and interference levels attained by HMM MAP estimators that possess these correlation models apriori.
\end{abstract}
\begin{IEEEkeywords}
Hidden Markov Model, Cognitive Radio, Spectrum Sensing, POMDP
\end{IEEEkeywords}
\vspace{-5mm}
\section{Introduction}\label{I}
The advent of fifth-generation wireless communication networks has exacerbated the problem of spectrum scarcity \cite{7158089}. Cognitive radio networks facilitate efficient spectrum utilization by intelligently accessing \emph{white spaces} left unused by the sparse and infrequent transmissions of licensed users, while ensuring rigorous incumbent non-interference compliance \cite{4562537}. 

A crucial aspect underlying the design of cognitive radio networks is the ability to perform spectrum sensing. However, physical design limitations are imposed on the cognitive radio's spectrum sensor in view of quick turnaround times and energy efficiency \cite{5990482}, which restrict the number of channels that can be sensed at any given time. This has led to research in algorithms that first determine the best channels to sense, after which the gathered information is used to perform channel access. The state-of-the-art are based on multi-armed bandits \cite{7094730}, reinforcement learning \cite{6507570}, and custom heuristics \cite{4554696, 6956794}. However, most of these works, such as \cite{7094730, 6507570, 7895211, 7336513, 8571293}, assume independence across frequency bands, which is imprudent because licensed users may exhibit correlation across both frequency and time in their channel occupancy behavior: they may occupy a set of adjacent frequency bands for an extended period of time \cite{6188346}. This correlation structure may be leveraged for more accurate predictions of spectrum holes. In this paper, we propose a parameter estimation algorithm to learn the frequency and time correlation structure, based on which we solve for the optimal sensing and access policy.

Distributed spectrum sensing has been considered in \cite{6507570} and solved using SARSA with linear value function approximation. However, frequency correlation is precluded, and errors in state estimation are neglected in the decision process. In \cite{6956794}, the frequency correlation is exploited, but a noise-free observation model is assumed. Compared to \cite{6507570, 6956794}, we account for the uncertainty in the occupancy state and for noisy observations via a partially observable Markov decision process (POMDP) formulation. Standard MAP-based state estimators for Hidden Markov Models (HMMs) such as the Viterbi algorithm can be employed to estimate spectrum occupancy \cite{4554696}; however, these estimators rely on knowledge of the transition model, which may be unknown in practice. Therefore, in our paper, we embed a parameter estimation algorithm to learn the parametric time-frequency correlation models. Additionally, \cite{4554696} does not impose sensing restrictions on the cognitive radio. Finally, in \cite{6956794, 4554696}, the time-frequency correlation structure is estimated offline based on pre-loaded databases. Instead, in our work, we present a fully online framework to estimate it, and simultaneously, solve for the optimal sensing and access policy.

The contributions of this paper are as follows: a POMDP formulation detailing the optimization problem for spectrum sensing and access in a radio environment with multiple licensed users exhibiting correlations in their occupancy behavior across both time and frequency, assuming a linear, Gaussian observation model with sensing limitations; an online parameter estimation algorithm to learn these correlation models; and a concurrent randomised point-based value iteration algorithm that solves the POMDP formulation for the optimal spectrum sensing and access policy. The rest of the paper is organized as follows: in Sec. \ref{II}, we define the system model, followed by the formulations, approaches, and algorithms in Sec. \ref{III}; in Sec. \ref{IV}, we present numerical evaluations, followed by our conclusions in Sec. \ref{V}.
\vspace{-4mm}
\section{System Model}\label{II}
\noindent {\bf Signal Model:}
We consider a network consisting of $J$ licensed users termed the Primary Users (PUs) and one cognitive radio termed the Secondary User (SU) equipped with a spectrum sensor. The objective of the SU is to opportunistically access portions of the spectrum left unused by the PUs in order to maximize its own throughput. To this end, the SU should learn how to intelligently access spectrum holes (white spaces) intending to maximize its throughput while maintaining strict non-interference compliance with incumbent transmissions. The observed wideband signal in the frequency domain is given by
\begin{equation}\label{2}
    Y_k(i) = \sum_{j=1}^{J} H_{j,k}(i)X_{j,k}(i) + V_k(i),
\end{equation}
where $i {\in} \{1,2,3,\dots,T\}$ represents the time index; $k {\in} \{1,2,3,\dots,K\}$ represents the index of the components in the frequency domain; $V_k(i) {\sim} \mathcal{CN}(0,\sigma_V^2)$ represents circularly symmetric additive complex Gaussian noise, i.i.d across frequency and across time, and independent of $H$ and $X$; $X_{j,k}(i)$ is the signal of the $j$th PU in the frequency domain, and $H_{j,k}(i)$ is its frequency domain channel. We further assume that the $J$ PUs employ an orthogonal access to the spectrum (e.g., OFDMA) so that $X_{j,k}(i)X_{g,k}(i)=0, \forall j{\neq}g$. Thus, letting $j_{k}$ be the index of the PU that contributes to the signal in the $k$th spectrum band, and letting  $H_{k}(i){=}H_{j_{k},k}(i)$ and $X_{k}(i){=}X_{j_{k},k}(i)$ (with $X_{k}(i){=}0$ if no PU is transmitting in the $k$th spectrum band at time $i$), we can rewrite \eqref{2} as 
\begin{equation}\label{3}
    Y_k(i) = H_{k}(i)X_{k}(i) + V_k(i).
\end{equation}
We model $H_{k}(i)$ as a zero-mean circularly symmetric complex Gaussian random variable with variance $\sigma_H^2$, $H_k {\sim} \mathcal{CN}(0,\sigma_H^2)$, i.i.d. across frequency bands, over time, and independent of the occupancy state of the channels.
\begin{figure}
    \centering
    \includegraphics[width=0.80\linewidth]{MarkovChainsVisualization}
    \vspace{-5mm}
    \caption{The correlation model across time and frequencies underlying the occupancy behavior of incumbents in the network}
    \vspace{-5mm}
    \label{fig:1}
\end{figure}
%\vspace{-3mm}

\noindent {\bf PU Spectrum Occupancy Model:}
We now introduce the model of PU occupancy over time and across the frequency domain. We model each $X_k(i)$ as 
\begin{equation}\label{4}
    X_k(i) = \sqrt{P_{tx}}B_k(i)S_k(i),
\end{equation}
where $P_{tx}$ is the transmission power of the PUs, $S_k(i)$ is the transmitted symbol modelled as a constant amplitude signal, $|S_k(i)|{=}1$, i.i.d. over time and across frequency bands;\footnote{In the case where $S_k(i)$ does not have constant amplitude, we may approximate $H_{k}(i)S_{k}(i)$ as complex Gaussian with zero mean and variance $\sigma_H^2\mathbb E[|S_{k}(i)|^2]$, without any modification to the subsequent analysis.} $B_k(i){\in}\{0,1\}$ is the binary spectrum occupancy variable, with $B_k(i){=}1$ if the $k$th spectrum band is occupied by a PU at time $i$, and $B_k(i){=}0$ otherwise. Therefore, the PU occupancy behavior in the entire wideband spectrum of interest at time $i$, discretized into narrow-band frequency components can be modeled as the vector 
\begin{equation}\label{5}
    \vec{B}(i) = [B_1(i), B_2(i), B_3(i), \cdots, B_K(i)]^T {\in} \{0, 1\}^K.
\end{equation}
PUs join and leave the spectrum at random times. To capture this temporal correlation in the spectrum occupancy dynamics of PUs, we model $\vec{B}(i)$ as a Markov process,
\begin{equation}\label{6}
    \begin{aligned}
        \mathbb{P}(\vec{B}(i+1)|\vec{B}(j), \forall j \leq i) = \mathbb{P}(\vec{B}(i+1)|\vec{B}(i)).
    \end{aligned}
\end{equation}
Additionally, when joining the spectrum pool, PUs occupy a number of adjacent spectrum bands, and may vary their spectrum needs depending on traffic demands, channel conditions, etc. To capture this behavior, we model $\vec{B}(i)$ as having Markovian correlation across spectrum bands,
\begin{align}\label{7}
&         \mathbb{P}(\vec{B}(i+1)|\vec{B}(i))\\&=
\nonumber
         \mathbb{P}(B_{1}(i+1)|B_{1}(i))
         \prod_{k=2}^{K} \mathbb{P}(B_{k}(i+1)|B_{k}(i), B_{k-1}(i+1)).
\end{align}
That is, the spectrum occupancy at time $i+1$ in frequency band $k$, $B_{k}(i+1)$, depends on the  occupancy state of the adjacent spectrum band at the same time, $B_{k-1}(i+1)$, and that of the same spectrum band $k$ in the previous time index $i$, $B_{k}(i)$ as shown in Fig. \ref{fig:1}. We structure the correlation models as two Markov chains: one across time and the other across frequencies, where the chain across frequencies is parameterized by $p{=}\mathbb{P}(B_{k}(i+1){=}1|B_{k-1}(i+1){=}0)$ and the chain across time is parameterized by $q{=}\mathbb{P}(B_{k}(i+1){=}1|B_{k}(i){=}0)$. We estimate these parameters $p$ and $q$, parameterizing each of these two chains, using the parameter estimation algorithm described in Sec. \ref{III} in order to obtain the transition model underlying the MDP, given by \eqref{7}.
% \vspace{-3mm}

\noindent{\bf Spectrum Sensing Model:}
In order to detect the available spectrum holes, the SU performs spectrum sensing. However, owing to physical design limitations at the SU's spectrum sensor, the SU can sense only $\kappa$ out of $K$ spectrum bands at any given time, with $1{\leq}\kappa{\leq}K$. Let $\mathcal K_{i}{\subseteq}\{1,2,\dots,K\}$ with $|\mathcal K_i|{\leq}\kappa$ be the set of indices of spectrum bands sensed by the SU at time $i$, which is part of our design.
Then, we define the observation vector
\begin{equation}\label{8}
    \vec{Y}(i) = [Y_k(i)]_{k {\in} \mathcal K_i},
\end{equation}
where $Y_k(i)$ is given by \eqref{3}.
The true states $\vec{B}(i)$ encapsulate the actual occupancy behavior of the PU and the measurements at the SU are noisy observations of these true states which are modeled to be the observed states of an HMM. Given $\vec{B}(i)$ and $\mathcal K_i$, the probability density function of $\vec{Y}(i)$ is
\begin{equation}\label{9}
    f(\vec{Y}(i)|\vec{B}(i), \mathcal K_i) = \prod_{k \in \mathcal K_i} f(Y_k(i)|B_k(i)),
\end{equation}
owing to the independence of channels (given the occupancy states), noise, and transmitted symbols across frequency bands. Moreover, from \eqref{3},
\begin{equation}\label{10}
 Y_k(i)|B_k(i) \sim \mathcal{CN}(0, \sigma_H^2P_{tx}B_k(i) + \sigma_V^2).
\end{equation}

\noindent{\bf POMDP Agent Model:}
In this section, we model the spectrum access scheme of the SU as a POMDP, whose goal is to devise an optimal sensing and access policy in order to maximize its throughput while maintaining strict non-interference compliance with incumbent transmissions. In fact, the agent's limited sensing capabilities coupled with its noisy observations result in an increased level of uncertainty at the agent's end about the occupancy state of the spectrum under consideration and the exact effect of executing an action on the radio environment. The transition model of the underlying MDP as described by \eqref{7}, is denoted by $\mathbf{A}$ and is learned by the agent by interacting with the radio environment (see Sec. \ref{III}). The emission model is denoted by $\mathbf{M}$ and is given by \eqref{9}, with $f(Y_k(i)|B_k(i))$ given by \eqref{10}. 

We model the POMDP as a tuple $(\mathcal B,\mathcal{A},\mathcal{Y},\mathbf{A},\mathbf{M})$ where $\mathcal{B}\equiv\{0,1\}^K$ represents the state space of the underlying MDP with states $\vec{B}$, given by all possible realizations of the spectrum occupancy vector as described by \eqref{5}; $\mathcal{A}$ represents the action space of the agent, given by all possible combinations in which $\kappa$ spectrum bands are chosen to be sensed out of $K$ at any given time; and $\mathcal{Y}$ represents the observation space of the agent based on the aforementioned signal model. The state of the POMDP at time $i$ is given by the \emph{prior belief} $\beta_i$, which represents the probability distribution of the underlying MDP state $\vec{B}(i)$, given the information collected by the agent up to time $i$, but before collecting the new information in time-step $i$. At the beginning of each time index $i$, given $\beta_i$, the agent selects $\kappa$ spectrum bands out of $K$, according to a policy $\pi(\beta_i)$, thus defining the sensing set $\mathcal K_i$, performs spectrum sensing  on these spectrum bands, observes $\vec{Y}(i){\in} \mathcal{Y}$, and updates its \emph{posterior belief} $\hat{\beta}_i$ of the current spectrum occupancy $\vec{B}(i)$ as 
\begin{align}\label{11}
\hat\beta_i(\vec{B}') &= \mathbb{P}(\vec{B}(i) = \vec{B}'|\vec{Y}(i), \mathcal K_i, \beta_i)\\&=
\nonumber
\frac{\mathbb{P}(\vec{Y}(i)|\vec{B}', \mathcal{K}_i) \beta_i(\vec{B}')}{
\sum_{\vec{B}'' {\in} \{0,1\}^K} \mathbb{P}(\vec{Y}(i)|\vec{B}'', \mathcal{K}_i) \beta_i(\vec{B}'')}.
\end{align}
We denote the function that maps the prior belief $\beta_i$ to the posterior belief $\hat\beta_i$ through the spectrum sensing action $\mathcal K_i$ and the observation signal $\vec{Y}(i)$ as $\hat\beta_i{=}\hat{\mathbb B}(\beta_i, \mathcal K_i, \vec{Y}(i))$.

Given the posterior belief $\hat{\beta}_i$, we estimate the occupancy state of the discretized spectrum under consideration as $\vec{B}(i)^{*}{=}\argmax_{\vec{B} {\in} \mathcal{B}} \hat{\beta}_{i}(\vec{B})$. Let $B_{k}(i)^{*}{=}\phi_{k}(\hat{\beta}_{i}) {\in} \{0, 1\}$ be the estimated state of channel $k$ at time $i$. If the channel is deemed to be idle as a result of this MAP estimation procedure, i.e., $\phi_{k}(\hat{\beta}_{i}){=}0$, the SU accesses the channel for delivering its network flows. Else, it leaves it untouched. Given the PU occupancy state $\vec{B}(i)$ and posterior belief $\hat\beta_i$, the reward metric of the POMDP is given by the number of \emph{truly idle} bands detected by the SU accounting for the throughput maximization aspect of the agent's objective and a penalty for \emph{missed detections} accounting for the incumbent non-interference constraint, i.e.,
\begin{equation}
\nonumber
    R(\vec{B}(i), \hat{\beta}_i){=}\sum_{k=1}^{K} (1{-}B_k(i))(1{-}\phi_k(\hat{\beta}_{i})){-}\lambda B_k(i)(1 - \phi_k(\hat{\beta}_i)),
\end{equation}
where $\lambda{>}0$ represents a penalty factor. After performing data transmission, the SU computes the prior belief for the next time-step based on the dynamics of the Markov chain as
\begin{equation}\label{13}
    \beta_{i+1}(\vec{B}') = \mathbb{P}(\vec{B}(i+1) = \vec{B}'|\hat{\beta}_{i}).
\end{equation}
We denote the function that maps the posterior belief $\hat\beta_i$ to the prior belief $\beta_{i+1}$ as $\beta_{i+1}{=}{\mathbb B}(\hat\beta_i)$.
The goal of the problem at hand is to determine an optimal spectrum sensing policy to maximize the infinite-horizon discounted reward,
\begin{equation}\label{14}
    \pi^{*}{=}\argmax_{\pi} V^{\pi}(\beta) \triangleq \mathbb{E}_{\pi} \Big[\sum_{i=1}^{\infty} \gamma^{i} R(\vec{B}(i), \hat{\beta}_i)|\beta_0 {=}\beta\Big],
\end{equation}
where $0{<}\gamma{<}1$ is the discount factor, $\beta_0$ is the initial belief, and $\hat\beta_i$ is the posterior belief induced by policy $\mathcal K_i{=}\pi(\beta_i)$ and the observation $\vec{Y}(i)$ via $\hat\beta_i{=}\hat{\mathbb B}(\beta_i, \mathcal K_i, \vec{Y}(i))$, and we have defined the value function $V^{\pi}(\beta)$ under policy $\pi$ starting from belief $\beta$.
The optimal policy $\pi^*$ and the corresponding optimal reward $V^*(\beta)$ are the solutions of Bellman's optimality equation $V^*{=}H[V^*]$, where the operator $V_{n+1}{=}H[V_n]$ is defined as
\begin{align}\label{16}
\nonumber
        V_{n+1}(\beta) = &\max_{\mathcal{K} {\in} \mathcal{A}} \sum_{\vec{B} {\in} \mathcal{B}} \beta(\vec{B}) \mathbb{E}_{\vec{Y}|\vec{B}, \mathcal{K}} \Big[R(\vec{B}, \hat{\mathbb{B}}(\beta, \mathcal{K}, \vec{Y}))\\ &+\gamma V_n(\mathbb{B}(\hat{\mathbb{B}}(\beta, \mathcal{K}, \vec{Y})))\Big],\ \forall \beta.
\end{align}

This problem can be solved using the value iteration algorithm, i.e., by solving \eqref{16} iteratively until convergence to a fixed point. However, given the high dimensionality of the spectrum sensing and access problem, i.e., the number of states of the underlying MDP scales exponentially with the number of bands in the spectrum, solving equation \eqref{16} using Exact Value Iteration and Policy Iteration algorithms is computationally infeasible. Additionally, solving for the optimal policy from equation \eqref{16} requires prior knowledge about the underlying MDP's transition model. Therefore, in this paper, we present a framework to estimate the transition model of the underlying MDP online, while concurrently utilizing this learned model to solve for the optimal policy by employing randomized point-based value iteration techniques, namely, the PERSEUS algorithm \cite{DBLP:journals/corr/abs-1109-2145}.
\vspace{-3mm}
\section{Approaches and Algorithms}\label{III}
\noindent{\bf Occupancy Behavior Transition Model Estimation:}
In real-world implementations of cognitive radio systems, the transition model of the occupancy behavior of the PUs is unknown to the SUs in the network and needs to be learned over time. The learned model then needs to be fed back to the POMDP agent which is solving for the optimal spectrum sensing and access policy simultaneously. Inherently, the approach constitutes solving the Maximum Likelihood Estimation (MLE) problem
\begin{equation}\label{17}
    \vec{\theta}^{*} = \argmax_{\vec{\theta}} \mathbb{P}([\vec{Y}(i)]_{i=1}^{\tau}|\vec{\theta}),
\end{equation}
where $\vec{\theta}{=}[p\ q]^{T}$ and $\tau$ refers to the learning period of the parameter estimator: this can be equal to the entire duration of the POMDP agent's interaction with the radio environment implying simultaneous model learning or can be a predefined parameter learning period before triggering the POMDP agent. In order to facilitate better readability, for the description of this parameter estimator, we denote $[\vec{Y}(i)]_{i=1}^{\tau}$ as $\mathbf{Y}$ and $[\vec{B}(i)]_{i=1}^{\tau}$ as $\mathbf{B}$. Re-framing \eqref{17} as an optimization of the log-likelihood, we get,
\begin{equation}\label{18}
    \vec{\theta}^{*} = \argmax_{\vec{\theta}} \log\Big(\sum_{\mathbf{B}} \mathbb{P}(\mathbf{B}, \mathbf{Y}| \vec{\theta})\Big).
\end{equation}
This problem can be solved using the Expectation-Maximization (EM) algorithm \cite{778361}, where the E-step constitutes
\begin{equation}
    Q(\vec{\theta}|\hat{\vec{\theta}}^{(t)}) = \mathbb{E}_{\mathbf{B}|\mathbf{Y}, \hat{\vec{\theta}}^{(t)}} \Big[ \log \Big(\sum_{\mathbf{B}} \mathbb{P}(\mathbf{B}, \mathbf{Y}|\hat{\vec{\theta}}^{(t)}) \Big) \Big],
\end{equation}
which can be obtained by employing the Forward-Backward algorithm using the current estimate of $\vec{\theta}$, i.e., $\vec{\theta}^{(t)}$ \cite{778361}, and the M-step constitutes
\begin{equation}
    \hat{\vec{\theta}}^{(t+1)} = \argmax_{\vec{\theta}} Q(\vec{\theta}|\hat{\vec{\theta}}^{(t)}),
\end{equation}
which involves re-estimation of the maximum likelihood parameters in $\vec{\theta}$ using the statistics obtained from the Forward-Backward algorithm.
%\vspace{-2mm}

\noindent{\bf The PERSEUS Algorithm:}
We solve for the optimal spectrum sensing and access policy, formulated as a POMDP, in parallel with the parameter estimation algorithm, employing the model estimates until the EM algorithm converges; after which, we utilize this converged transition model until the POMDP value iteration algorithm converges. As discussed in Sec. \ref{II} of this article, solving the Bellman equation \eqref{16} for POMDPs with large state and action spaces using exact value iteration and policy iteration techniques is computationally infeasible \cite{DBLP:journals/corr/abs-1109-2145}. Hence, we resort to approximate value iteration techniques to ensure that the system scales well to a large number of bands in the spectrum of interest. One such technique, the PERSEUS algorithm \cite{DBLP:journals/corr/abs-1109-2145} is a randomized point-based approximate value iteration method that involves an initial phase of determining a set of so-called \emph{reachable beliefs} $\tilde{\mathcal{B}}$ by allowing the agent to randomly interact with the radio environment. The goal of the PERSEUS algorithm is to improve the value of all the belief points in this set $\tilde{\mathcal{B}}$ by updating the value of only a subset of these belief points, chosen iteratively at random. Using the notion that, for infinite-horizon POMDPs, $V^*$ in \eqref{16} can be approximated by a Piece-Wise Linear and Convex function (PWLC) \cite{DBLP:journals/corr/abs-1109-2145}, the PERSEUS algorithm operates on the core idea that the value function at time index $i$ can be parameterized by a set of hyperplanes $\{\vec{\alpha}_i^{u}\}$, $u {\in} \{1,2,\dots,|\tilde{\mathcal{B}}|\}$, each of which represents a region of the belief space for which it is the maximizing element. The belief points in $\tilde{\mathcal{B}}$ are to be improved over numerous iterative \emph{backup} stages. The optimal hyperplane $\vec{\alpha}_{i+1}^{u}$ at time index $i{+}1$ for the $u$th belief $\beta_{u} \in \tilde{\mathcal{B}}$ can be iteratively computed as, from \cite{DBLP:journals/corr/abs-1109-2145},
\begin{equation}\label{20}
    \vec{\alpha}_{i+1}^{u} = \text{backup}(\beta_{u}){=}\argmax_{\mathcal{K} \in \mathcal{A}} \beta_u \cdot \Xi_{\mathcal{K}}^{u},
\end{equation}
where $\beta\cdot\Xi{=}\sum_{\vec{B}}\beta(\vec{B})\Xi(\vec{B})$ denotes inner product. $\Xi_{\mathcal{K}}^{u}$ is the hyperplane corresponding to a one-step look-ahead of the value iteration updates under action $\mathcal K$ and belief $\beta_u$, given by
\begin{equation}
        \Xi_{\mathcal{K}}^{u} = \sum_{\vec{B} {\in} \mathcal{B}} \beta_{u}(\vec{B}) \mathbb{E}_{\vec{Y}|\vec{B}, \mathcal{K}} \Big[R(\vec{B}, \hat{\mathbb{B}}(\beta_{u}, \mathcal{K}, \vec{Y}))+\gamma 
        \Xi_{\mathcal{K}, \vec{Y}}^{u}(\vec{B})\Big],
\nonumber
\end{equation}
where $\Xi_{\mathcal{K}, \vec{Y}}^{u}$ is the hyperplane associated with the future value function, computed from the previous set of hyperplanes as
\begin{equation}
    \Xi_{\mathcal{K}, \vec{Y}}^{u}=\argmax_{u' {\in} \{1, 2, \dots, |\tilde{\mathcal{B}}|\}} \mathbb{B}(\hat{\mathbb{B}}(\beta_{u}, \mathcal{K}, \vec{Y}))\cdot\alpha_{i}^{u'}.
\nonumber
\end{equation}
Once these hyperplanes have been computed, the new value function at a generic belief $\beta$ and the corresponding policy can be computed as
\begin{equation}\label{40}
V_{i+1}(\beta) =\!\!\!\!\!\!\!\!\max_{u {\in} \{1, 2, \dots, |\tilde{\mathcal{B}}|\}}\! \beta \cdot \vec{\alpha}_{i+1}^u,
\ \pi_{i+1}(\beta) = a(\vec{\alpha}_{i+1}^{*}),\!\!
\end{equation}
where $a(\vec{\alpha}_{i+1}^{*})$ is the action corresponding to the maximizing hyperplane $\vec{\alpha}_{i+1}^{*}$. In each backup stage, the agent samples a belief $\beta$ uniformly at random from the set of unimproved points and performs a backup on this sampled belief point according to \eqref{20}, to determine the optimal hyperplane. Considering an arbitrary time index $i{+}1$, if $V_{i+1}(\beta){=}\beta{\cdot}\vec{\alpha}{\geq}V_{i}(\beta)$, then the belief point $\beta$ is said to be improved along with any other belief points $\beta'$ in the unimproved set for which $V_{i+1}(\beta'){=}\beta'{\cdot}\vec{\alpha}{\geq}V_{i}(\beta')$. If $V_{i+1}(\beta){=}\beta{\cdot}\vec{\alpha}{<}V_{i}(\beta)$, then a copy of the maximizing hyperplane for $V_i(\beta)$ is used for $V_{i+1}(\beta)$ and the belief point $\beta$ is then removed from the set of unimproved points. The backup stage continues until the set of unimproved points is empty and the agent performs a series of backup stages until the number of policy changes between consecutive iterations is below a specified threshold $\eta$. The belief update procedure outlined in \eqref{11} is an essential aspect of the PERSEUS algorithm which can turn into a performance bottleneck for large state spaces due to the inherent iteration over all possible states. In order to circumvent this problem, we fragment the spectrum into much smaller, independent sets of correlated channels and then run the PERSEUS algorithm on these fragments by leveraging multi-processing and multi-threading tools available at our disposal in software frameworks. Furthermore, we avoid iterating over all possible states and allow only those state transitions we deem to be the most probable - for example, we allow only those state transitions that involve a Hamming distance of up to $3$ between the previous state vector and the current state vector in an $18$ channel radio environment.
\vspace{-3.5mm}
\section{Numerical Evaluations}\label{IV}
\noindent{\bf Simulation Setup:}
We simulate a radio environment with $3$ PUs occupying a set of $18$ channels, according to a Markovian time-frequency correlation structure defined by the parameters $p{=}q{=}0.3$, and an SU intelligently trying to access the available white spaces across both time and frequencies, with a channel model emulated using a noise variance of $\sigma_{V}^{2}{=}1$ and a channel impulse response variance of $\sigma_{H}^{2}{=}80$. Assuming that the SU is always backlogged, each channel offers a throughput of $1$ Mbps for the SU, when the channel is truly idle. We further assume that the SU accesses all channels that are deemed idle by the POMDP agent in any given time-step $i$, thereby giving us an SU throughput metric described as $\sum_{k=1}^{K}(1{-}B_{k}(i))(1{-}\phi_{k}(\hat{\beta}_{i}))$ Mbps. Additionally, we evaluate PU interference by defining an indicator variable $\mathcal{I}$ which is assigned a value of $1$ when the SINR at the PU, for a specific channel $k$, at a given time-step $i$, exceeds $15$dB, thereby giving us a total interference metric described as $\sum_{k=1}^{K}\mathcal{I}(\text{SINR}_{k}(i){<}15\text{dB})$. We impose a sensing restriction on the SU: only $6$ out of $18$ channels can be sensed by the SU in any given time-step $i$. Finally, while solving for the optimal policy in the PERSEUS algorithm, we employ a discount factor of $\gamma{=}0.9$.

\noindent{\bf Evaluations:}
The plot depicted in Fig. \ref{fig:4} shows the mean square error convergence of the parameter estimation algorithm while determining the parametric time-frequency correlation structure, i.e., $p$ and $q$. Starting with an initial estimate of $10^{-8}$, the EM algorithm detailed in Sec. \ref{III} converges to the true transition model with an error of $\epsilon{\leq}10^{-8}$ over numerous iterations, each iteration corresponding to an averaging operation constituting 300 observation vectors. We observe the mean square error given by $\mathbb{E}[(\theta_{i}{-}\hat{\theta}_{i}^{(t)})^{2}],\theta_{i}{\in}[p,q]^{\intercal}$ iteratively reduces as it goes through the E-step and the M-step. It has been theoretically shown to converge, i.e., each iteration either improves the true likelihood or leaves it unchanged \cite{Neal1998}. Since the EM algorithm is susceptible to premature convergence to local optima and saddle points, we mitigate this by averaging the procedure over several cycles.
\begin{figure}
    \centering
    \includegraphics[width=0.80\linewidth]{PerseusRegretConvergence_MeanSquareErrorConvergence.png}
    \caption{Mean square error convergence of the parameter estimation algorithm while determining the correlation models $p$ and $q$, and the Regret convergence of the fragmented PERSEUS algorithm with belief simplification}
    \vspace{-5mm}
    \label{fig:4}
\end{figure}

Fig. \ref{fig:4} also illustrates the \emph{Regret} convergence plot of the PERSEUS algorithm, on the same time scale and in the same simulation run, over several iterations $t$, wherein the regret metric corresponds to the difference in utility obtained by our PERSEUS algorithm at a certain iteration $t$ in our simulation, denoted by $R_{P}^{(t)}(\vec{B}(i), \hat{\beta}_{i})$, and an \emph{Oracle} which has complete information about the occupancy behavior of incumbents in the network, whose utility is denoted by $R_{O}(\vec{B}(i))$. Starting with a random exploration phase to gather the set of reachable beliefs $\tilde{\mathcal{B}}$, the termination condition for the PERSEUS algorithm is that the number of policy changes, denoted by $\eta$, over several consecutive backup stages should be $0$. This trace in Fig. \ref{fig:4}, similar to the \emph{Reward v/s Time} plot in \cite{DBLP:journals/corr/abs-1109-2145}, serves as a measure of convergence for our fragmented PERSEUS algorithm with simplified belief updates and online model estimation.

We evaluate the performance of the proposed framework in terms of the SU network throughput and PU interference metrics over varying values of the penalty term $\lambda$ as illustrated in Fig. \ref{fig:8}. As surmised, we find that our POMDP agent decides to limit channel access when the penalty is high, leading to lower SU network throughput and lower PU interference; and on the other hand, it follows a more lenient channel access strategy when the penalty is low, resulting in higher SU network throughput and higher PU interference. In general, we observe the trend of rising throughput and increasing interference as the penalty for missed detections $\lambda$ is lowered. Comparing this performance of our proposed framework with correlation-coefficient based state-of-the-art, namely the MEM with MEI-CCE and MPE algorithm with $\rho_{th}{=}0.77$ and $6$ specified clusters, from \cite{6956794}, we find that our framework achieves higher SU network throughput and lower PU interference with $\lambda{\geq}10$. Furthermore, the proposed framework comes very close to achieving the throughput attained by a Viterbi agent \cite{4554696}, while providing the same interference performance. It is worth noting that the Viterbi agent possesses prior knowledge about the transition model of the underlying MDP and senses more channels per time-step than our POMDP agent. More importantly, the proposed framework allows us to regulate the trade-off between the interference caused to PUs and the throughput of the SU, by adjusting the parameter $\lambda$.
\vspace{-3mm}
\section{Conclusion}\label{V}
\begin{figure}
    \centering
    \includegraphics[width=0.80\linewidth]{SU_Throughput_PU_Interference_Varying_Penalty.png}
    \caption{SU Network Throughput versus PU Interference evaluation of the proposed framework over varying values of the penalty $\lambda$}
    \vspace{-5mm}
    \label{fig:8}
\end{figure}
In this paper, we formulate the optimal spectrum sensing and access problem in an AWGN observation model with multiple licensed users and a cognitive radio restricted in terms of its sensing capabilities, as a POMDP. In a radio environment wherein the occupancy behavior of the incumbents is correlated across time and frequencies, we present a consolidated framework that employs the EM algorithm to estimate the transition model of this occupancy behavior, and leverage a fragmented PERSEUS algorithm with belief update heuristics to simultaneously solve for the optimal spectrum sensing and access policy. Through system simulations, we conclude that our framework, in terms of the trade-off between secondary network throughput and interference to licensed users, out-performs the existing correlation-coefficient based clustering algorithms and a Neyman-Pearson detector that assumes independence among channels. Our framework is also capable of achieving the SU throughput and PU interference performance attained by a Viterbi algorithm that senses more channels per time-step, and that possesses the transition model information \emph{apriori}.
\bibliographystyle{IEEEtran}
\vspace{-5.5mm}
\bibliography{ref}
\end{document}